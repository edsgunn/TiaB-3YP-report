% Chapter Template

\chapter{Model} % Main chapter title

\label{Chapter4} % Change X to a consecutive number; for referencing this chapter elsewhere, use \ref{ChapterX}

%----------------------------------------------------------------------------------------
%	SECTION 1
%----------------------------------------------------------------------------------------

\section{Introduction}


The problem of generation of a set of chords from a melody is very similar to that of translating one language to another. 
The translation problem is one that is very popular in machine learning research and thus there is many resources on it.  
However the music related problem is harder to solve due to the extra dimension each of its elements contains. 
Each element in a melody has both pitch, represented by a descrete symbol or note, and duration whereas each element in the sequence of language is composed of only the discrete symbols or letters.  
Therefore in order to use techniques developed for natural language processing it is necessary to encode the melody and chords in such a way that their dimensions are collapsed into one. 

\section{Related Work}

MySong \\
BLSTMS guy \\
ML in Automatic Chord Generation \\
SeqGAN \\  
MuseGAN \\      
ChordGAN \\       
CLSTMGAN for melody Generation \\

\section{Model Requirements}
\subsection{Real Requrements}

\subsection{MVP Requrements}

\section{Models}

\subsection{GANs}

\subsection{RNNs}

\subsection{LSTMs}

\subsection{Transformers}

\section{Training}

\subsection{Issues}

\section{Results}