% This chapter with be all the B2 stuff as well as the introduction and definiton of the project

\chapter{Project Definition} % Main chapter title

\label{Chapter1} % Change X to a consecutive number; for referencing this chapter elsewhere, use \ref{ChapterX}

%----------------------------------------------------------------------------------------
%	SECTION 1
%----------------------------------------------------------------------------------------

\section{Introduction}

The process of writing music often begins with and idea for a melody which is followed by the development of chords to accompany it.
The matching of chords to a melody is a skill which usually requires years of training in musical techniques such as harminisation.
There is a large market of amateur musicians who lack the necessary training for this task but would otherwise enjoy the experience of developing music.  
In this report we will detail the design of \textbf{INSERT NAME HERE}, a system which facilitates the generation of an appropriately matched set of chords to a given monophonic melody.
Users are able to record a melody using their microphone and regenerate the chord sequence until they feel the they have found one suitable for the intended feel of their song.
Songs and generated chord accompaniments can be played back to the user, saved to a library of songs and shared using our songwriting community feature. \\
The problem of converting a recorded melody to a set of chords can be decomposed into a set of simpler sub-probblems, these being:
The conversion of the recorded melody into a form which numerically represents its features. 
The generation of a set of chords from this numerical representation.
The latter of these two problems can be solved in many ways many of which require a detailed knowlege of music to find patterns in melodies to match to chords.  
\textbf{WHY IS THIS A PROBELM FOR US}
A method which requires little knowlege of music is the used of a machine learning model to extract these patters from data of known chord melody pairings from professionally composed music. 
The curation and processing of an appropriate dataset requires significant effort and care in itself.
This leads to a natural division of labour across the project into the catagories of:
User Interface and general product design - detailed in \ref{Chapter2} by Di Wan;
Curation and preparation of a dataset in an appropriate format - detailed in \ref{Chapter3} by Terence Tan;
The design of an appropriate machine learning model - detailed in \ref{Chapter4} by Edward Gunn;
The conversion of recorded melody to the same format as expressed in the dataset - detailed in \ref{Chapter5} by Kitty Fung.


%-----------------------------------
%	SUBSECTION 1
%-----------------------------------
\subsection{Musical terminology}

Throught this report we will use a variety of musical terminoligy which will be defined here. A piece of music is composed of a sequence of adjacent \textbf{measures}, periods of time in which notes and chords can be played. 
We will generally take a \textbf{measure} to mean a bar in the music, however it is not restricted to this. We restrict \textbf{chord} to refer to a triad of notes, the justification for this is explained in \textbf{reference to explanation}.
The use of \textbf{melody} refers to a monophonic melody in which only one note is played at a time, excluding accompanying chords, unless otherwise stated.
%-----------------------------------
%	SUBSECTION 2
%-----------------------------------

\subsection{Subsection 2}
Morbi rutrum odio eget arcu adipiscing sodales. Aenean et purus a est pulvinar pellentesque. Cras in elit neque, quis varius elit. Phasellus fringilla, nibh eu tempus venenatis, dolor elit posuere quam, quis adipiscing urna leo nec orci. Sed nec nulla auctor odio aliquet consequat. Ut nec nulla in ante ullamcorper aliquam at sed dolor. Phasellus fermentum magna in augue gravida cursus. Cras sed pretium lorem. Pellentesque eget ornare odio. Proin accumsan, massa viverra cursus pharetra, ipsum nisi lobortis velit, a malesuada dolor lorem eu neque.

%----------------------------------------------------------------------------------------
%	SECTION 2
%----------------------------------------------------------------------------------------

\section{Project management}

Sed ullamcorper quam eu nisl interdum at interdum enim egestas. Aliquam placerat justo sed lectus lobortis ut porta nisl porttitor. Vestibulum mi dolor, lacinia molestie gravida at, tempus vitae ligula. Donec eget quam sapien, in viverra eros. Donec pellentesque justo a massa fringilla non vestibulum metus vestibulum. Vestibulum in orci quis felis tempor lacinia. Vivamus ornare ultrices facilisis. Ut hendrerit volutpat vulputate. Morbi condimentum venenatis augue, id porta ipsum vulputate in. Curabitur luctus tempus justo. Vestibulum risus lectus, adipiscing nec condimentum quis, condimentum nec nisl. Aliquam dictum sagittis velit sed iaculis. Morbi tristique augue sit amet nulla pulvinar id facilisis ligula mollis. Nam elit libero, tincidunt ut aliquam at, molestie in quam. Aenean rhoncus vehicula hendrerit.


%----------------------------------------------------------------------------------------
%	SECTION 3
%----------------------------------------------------------------------------------------
\section{Tech strat}

%----------------------------------------------------------------------------------------
%	SECTION 4
%----------------------------------------------------------------------------------------
\section{Risk}


%----------------------------------------------------------------------------------------
%	SECTION 5
%----------------------------------------------------------------------------------------
\section{Finance}

%----------------------------------------------------------------------------------------
%	SECTION 6
%----------------------------------------------------------------------------------------
\section{Ethics}
From the National Society of Professional Engineers Code of Ethics \cite{codeofethics}, Engineering
work impacts life directly, thus it is of the utmost importance to ensure service 
upholds honesty, impartiality, fairness, and equity, and must be dedicated to the protection of
the public health, safety, and welfare. Since our design interacts with users
through a mobile application, ethical concerns have arisen regarding the interactions between users 
as well as data handling on our end. In this section we will address issues related to data ethics 
and online hate crime.

\subsection{Data ethics}
Governmental organisations have published regulations \cite{EUdataregulations2018} and frameworks
\cite{framework} that are minimum standards for handling data ethically. The core values of data ethics 
is to treat data responsibly and righteously.
\linebreak
There are 3 overarching principles to adhere to but in short, we as the developer should treat our 
users\textsc{\char13} data as we would for our own.

\subsubsection{Transparency}
We should be clear and publish specifications of our project in a way that users can understand and access
easily. It is necessary to delineate a privacy policy and makes sure users read and agree to it before they use
our application and keep the policy at a prominent location, i.e. in the footer for webpage, in the developer
section of the app listing or on the sign-in interface of the app.

\linebreak
We would include the following items in our privacy policy:
\begin{enumerate}
    \item Company licence and registration reference number: registration on https://ico.org.uk/ is required
    before collecting data from the public
    \item Details and purpose of data collected: we will collect
    \begin{description}
        \item[Audio data - ] Audio signal and metadata to feed in our machine learning model to improve accuracy.
        \item[Networks and connections - ] for our interactive community features
        \item[Usage - ] Contents and functions that the users have viewed or engaged with, as well as the duration.
        \item[Device information - ] Device attributes like signal strength, battery levels, version of operating system
        \item[Cookie data - ] Both first- and third-party cookies are needed. For the first-party cookies, to personalise and optimise user
        experience, we would save their usage preference like language, dark or light background mode. As for the third-party cookies,
        they are collected since we will be displaying advertisements for standard users, and these cookies will not be placed for users
        who have subscribed to our platform.
    \end{description}
    \item Data retention policy: We will specify how long information will be kept and the procedure of disposing the data when an user deactivated
    his account or when the data is no longer necessary to collect. Although there is no limitation on data storage, we will have to act ethically and
    decide the timeframe based on the genuine motivation of retaining the data. Data should only be collected when it is vital in the context
    of app operation, thus we should not keep data just in case it is needed in the future.


account deletion (we should store data until it is not necessary to collect/ users deactivate account)
(c)	Access rights of different members of the team
(d)	Third parties that will have access to the data collected
(e)	Rights that users have over their data (users should have the right to request a copy of data provided/ request the organisation to delete the data/ object to the organisation’s data processing)
(f)	Notification of changes to privacy policy and ask users to review/ accept the revised policy (if needed)
(g)	Security standards (SSL/ encryption standards)
(h)	Contact information
\end{enumerate}
\subsubsection{Accountability}
We should process personal data responsibly and systematically and endeavour to reduce risks for individuals as 
well as mitigate social and ethical implications.\cite{principles} To be publicly accountable, 
effective governance and oversight mechanisms that can be exercised by the public are
necessary. Moreover, we have to ensure our app does no harm in any way (i.e. should not include sensitive topics/ as the app carries on, there will not be a chance for our users to online abuse others)


\subsubsection{Fairness}
 Collect only necessary data for the app to run




\subsection{Online hate crime}
%----------------------------------------------------------------------------------------
%	SECTION 7
%----------------------------------------------------------------------------------------
\section{Responsible Innovation}
Not only does research bring novelty and value to society, it may also bring forward ethical dilemmas, social transformations and adverse consequences to society
-	considers the role that new products, processes or business models have in society 
-	responsible approach towards innovation involves creating change that has positive impacts on society and the environment

Components of RRI:
(i)	Anticipation: visioning consequences of the research and innovation. It is to ensure that the consequences of undertaking the research are considered and reflected in the research design. Although in this project we are not creating physical innovation/ technology, we are building a platform that allows users to interact ad communicate with each other.
(ii)	Reflection: researchers should always reflect on the research question they are investigating, the type of data collected, the method used to analyse the data and the implications of the findings. They should also judge if the research is required. An organisation can force reflection through organisational process/ structures like a project advisory board/ review & quality assurance steps. Reflection allows researchers to review the project from a macroscopic point of view.
(iii)	Ethics: researchers should uphold integrity and prevent misconduct/ negligence. They should be responsible for both the undergoing research and behaviour.
(iv)	Gender & Equality: Since male engineers still dominate the industry, it is easy to be biased in a project design, i.e. models built suit male users better.
(v)	Open Access: Making research/ project publicly available to everyone so the whole society can be benefitted by reducing wasteful duplication, increasing transparency & reproducibility of results.
(vi)	Governance: Organisations should establish practices that foster RRI, i.e. 
•	Having transparent and reflective internal procedures
•	Promoting participatory governance
•	Fostering stakeholder engagement exercises
•	Encouraging future-oriented governance
•	Valuing responsiveness
(vii)	Public Engagement: Researchers should fix the motivation and suitable audience before any public engagement, i.e. improving musical literacy

%----------------------------------------------------------------------------------------
%	SECTION 8
%----------------------------------------------------------------------------------------
\section{Sustainability}
As technology advances, energy consumed by computers and digital personal devices is taking up a larger portion in worldwide enerconsumption. There are about 8,918,157,500 active mobile devices consuming about 2 kWh/year energy (https://www.digitalinformationworld.com/2020/02/the-global-energy-consumption-of-information-technologies-infographic.html)
While a lot of effort has been put into reducing the energy consumption from the hardware perspective, it is also prime to look at the software implementation of a program. Other than writing energy-efficient codes (I.e. optimised searching and sorting algorithms), the programming language chosen can make a substantial difference too.

From paper: https://greenlab.di.uminho.pt/wp-content/uploads/2017/10/sleFinal.pdf
When it comes to resources required for running a program, we can categorise into energy, time and memory size. The research shows there is no strong correlation between the 3 components. Since our application has no requirements on the computational time and memory size, we can solely focus on the energy consumption between different languages. Even though the energy needed to run different algorithms varies for different language, it is certain that compiled language requires less energy compared to virtual machine and interpreted languages thus it would make sense to use C/ Rust/ C++ to code.

Also, since we plan to employ a cloud network architecture and host most of the operation on cloud, that would add an enormous carbon footprint to our project too.

Carbon3IT estimates datacentres, including colocation facilities, account for at least 12\% of UK electricity consumption, or 41.11TWh a year; Cisco has previously forecast global cloud IP traffic to exceed 14.1 zettabytes (ZB) by the end of 2020; IDC’s Seagate sponsored Data Age 2025 report projects overall data growth of 30% a year to hit 175 ZB – with data stored of 7.5ZB, up from 1.1ZB in 2019. 

Other than looking for ways to reduce energy consumption, we can also take the initiative by offsetting carbon footprints. We can fund an equivalent carbon dioxide saving elsewhere, I.e. planting trees, solar/ wind power projects.


%----------------------------------------------------------------------------------------
%	SECTION 9
%----------------------------------------------------------------------------------------
\section{Legality}
