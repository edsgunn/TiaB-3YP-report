% This chapter with be all the B2 stuff as well as the introduction and definiton of the project

\chapter{Project Definition} % Main chapter title

\label{Chapter1} % Change X to a consecutive number; for referencing this chapter elsewhere, use \ref{ChapterX}

%----------------------------------------------------------------------------------------
%	SECTION 1
%----------------------------------------------------------------------------------------

\section{Introduction}

The process of writing music often begins with and idea for a melody which is followed by the development of chords to accompany it.
The matching of chords to a melody is a skill which usually requires years of training in musical techniques such as harminisation.
There is a large market of amateur musicians who lack the necessary training for this task but would otherwise enjoy the experience of developing music.  
In this report we will detail the design of \textbf{INSERT NAME HERE}, a system which facilitates the generation of an appropriately matched set of chords to a given monophonic melody.
Users are able to record a melody using their microphone and regenerate the chord sequence until they feel the they have found one suitable for the intended feel of their song.
Songs and generated chord accompaniments can be played back to the user, saved to a library of songs and shared using our songwriting community feature. \\
The problem of converting a recorded melody to a set of chords can be decomposed into a set of simpler sub-probblems, these being:
The conversion of the recorded melody into a form which numerically represents its features. 
The generation of a set of chords from this numerical representation.
The latter of these two problems can be solved in many ways many of which require a detailed knowlege of music to find patterns in melodies to match to chords.  
\textbf{WHY IS THIS A PROBELM FOR US}
A method which requires little knowlege of music is the used of a machine learning model to extract these patters from data of known chord melody pairings from professionally composed music. 
The curation and processing of an appropriate dataset requires significant effort and care in itself.
This leads to a natural division of labour across the project into the catagories of:
User Interface and general product design - detailed in \ref{Chapter2} by Di Wan;
Curation and preparation of a dataset in an appropriate format - detailed in \ref{Chapter3} by Terence Tan;
The design of an appropriate machine learning model - detailed in \ref{Chapter4} by Edward Gunn;
The conversion of recorded melody to the same format as expressed in the dataset - detailed in \ref{Chapter5} by Kitty Fung.


%-----------------------------------
%	SUBSECTION 1
%-----------------------------------
\subsection{Subsection 1}

Nunc posuere quam at lectus tristique eu ultrices augue venenatis. Vestibulum ante ipsum primis in faucibus orci luctus et ultrices posuere cubilia Curae; Aliquam erat volutpat. Vivamus sodales tortor eget quam adipiscing in vulputate ante ullamcorper. Sed eros ante, lacinia et sollicitudin et, aliquam sit amet augue. In hac habitasse platea dictumst.

%-----------------------------------
%	SUBSECTION 2
%-----------------------------------

\subsection{Subsection 2}
Morbi rutrum odio eget arcu adipiscing sodales. Aenean et purus a est pulvinar pellentesque. Cras in elit neque, quis varius elit. Phasellus fringilla, nibh eu tempus venenatis, dolor elit posuere quam, quis adipiscing urna leo nec orci. Sed nec nulla auctor odio aliquet consequat. Ut nec nulla in ante ullamcorper aliquam at sed dolor. Phasellus fermentum magna in augue gravida cursus. Cras sed pretium lorem. Pellentesque eget ornare odio. Proin accumsan, massa viverra cursus pharetra, ipsum nisi lobortis velit, a malesuada dolor lorem eu neque.

%----------------------------------------------------------------------------------------
%	SECTION 2
%----------------------------------------------------------------------------------------

\section{Main Section 2}

Sed ullamcorper quam eu nisl interdum at interdum enim egestas. Aliquam placerat justo sed lectus lobortis ut porta nisl porttitor. Vestibulum mi dolor, lacinia molestie gravida at, tempus vitae ligula. Donec eget quam sapien, in viverra eros. Donec pellentesque justo a massa fringilla non vestibulum metus vestibulum. Vestibulum in orci quis felis tempor lacinia. Vivamus ornare ultrices facilisis. Ut hendrerit volutpat vulputate. Morbi condimentum venenatis augue, id porta ipsum vulputate in. Curabitur luctus tempus justo. Vestibulum risus lectus, adipiscing nec condimentum quis, condimentum nec nisl. Aliquam dictum sagittis velit sed iaculis. Morbi tristique augue sit amet nulla pulvinar id facilisis ligula mollis. Nam elit libero, tincidunt ut aliquam at, molestie in quam. Aenean rhoncus vehicula hendrerit.