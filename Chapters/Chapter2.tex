% Chapter Template
\tocdata{toc}{Di Wan}



\chapter{Product Design} % Main chapter title

\label{Chapter2} % Change X to a consecutive number; for referencing this chapter elsewhere, use \ref{ChapterX}

\section{Introduction}
App design combines User interface (UI) and User Experience (UX); while UI is concerned about how the app pages look and feel, such as the fonts, colours and arrangements of icons, UX focuses on the functionality and usability. The best app design process comprises research, 
ideation, problem identification, design, feedback and problem evaluation. Currently, we only focus on the mobile app design or specifically IOS design based on Apple Platform and Android Design. 
%-----------------------------------------------------------------------------------------------------------------------------UX
\section{UX Design}
UX Design decides how someone will use an app and create a viable product. During the UX process, mobile app design ideas are generated and validated to ensure that all choices will work so that our app works. 
\\The quality of user experience is the crucial factor in measuring the quality of the design. Usually, it is the key to distinguishing a successful app design from an unsuccessful one. 
Customers are becoming pickier about which app to use and so quick to abandon the app they do not enjoy, so investing time and effort in creating a great user experience is essential.
\subsection{Minimum Viable Product (MVP)}
\label{MVP}
A minimum viable product, or MVP, is a product with enough features to attract early-adopter customers and validate a product idea early in the development cycle. To decide which parts belong to the MVP design, we use the MoSCoW method to represent all the features we want to include in our design, which are M (Must Have) o S (Should Have) C (Could Have) o W (Will not Have). 
The table separates the features for our MVP from the advanced functions that we want to include in our design.

\begin{table}[ht]
\centering
\begin{tabular}{ |c|c| } 
 \hline
\textbf{MUST-HAVE FEATURES FOR MVP} & \textbf{SHOULD/ COULD HAVE FEATURES}\\ 
 \hline
 Sign-up and sign-in with email & Sign-in with third party accounts \\ 
 \hline
 Upload Audio & Real-time generation \\ 
 \hline
 Chord output & Chord regeneration \\ 
 \hline
 Chord layout &  Export to PDF \\ 
 \hline
 Share posts & Message channel \\ 
 \hline
 Recommendation Engine &  Lock Chords when regenerating\\ 
 \hline
 Help and support &  Pitch Tracking\\ 
 \hline
 File storage& File backup \\ 
 \hline
 \end{tabular}
 \caption{MoSCoW Table}
 \centering
 \label{moscow}
 \end{table}
 
 \subsection{User Flow and Functionality}
 \textbf{User flows} are flowcharts that illustrate the movement or journey of a user through the app. User flow diagrams are indispensable in mastering user experience. 
 They allow us to understand how users interact with our app and their steps to complete a task or achieve a goal on our app, 
 which will help us create a superior user experience for the user and meet their needs more efficiently. 
 \\One of our user flow examples is shown in Figure:\ref{flowchartmain}
 We use a rounded rectangle to represent the termination, a diamond to represent the decision, a rectangle to represent the process, and an arrow to represent the flow direction.

\begin{figure}[ht]
\centering
\includegraphics[scale = 0.25]{MFpage.png}
\caption{User Flow Diagram for main function page}
\label{flowchartmain}
\end{figure}


\subsection*{Main Functionality}

\begin{itemize}

\item \textbf{Sign-up and Sign-in with Email}
\\From the start, our users can sign up with their email and log in with our app accounts. We want our users to create an account so that we can profile our users more easily.

\item \textbf{Sign-in with third-party Accounts}
\\ The purpose is to reduce our users' barriers to entering our app. OAuth (Open Authorisation) is an open standard for access delegation, and it is commonly used for internet users to grant websites or applications access to their information on other websites without giving them the passwords. We will include OAuth in our design to allow our users to sign in with Facebook and Google accounts since those are the most common social media accounts people have.

\item \textbf{Ask for song tag preference}
\\After our users sign in, they will be asked about the feature preferences, and we provide limited choices of answers they can choose from. The answers will be stored and fed to our recommendation engine (section \ref{Content Based Recommendation System}).

\item \textbf{Store and backup files on Cloud}
\\The purpose is to enable better synchronisation between devices and accessibility to the files. Since developing our own cloud drive is time-consuming and less secure than directly combining the existing cloud platform into our design. 
We will include CloudKit in our IOS design to allow users to store their saved files in iCloud.

\item \textbf{Recording input}
\\ To obtain the audio source, we allow our users to record their singing using our app or upload a previous recording 
from a built-in app such as Voice Memos on iPhone. To fit the audio processing (section:\ref{sec:crepe}), the recording will be initially sampled at 16hz.

\item \textbf{Chord generation and regeneration}
\\Unlike Chordify, which generates the same chords each time for the same audio source, we provide the option for users to regenerate the chords. We aim to provide sightly different chords when the regeneration button is pressed. 
We will also allow our users to lock the chords they like during the regeneration.

\item \textbf{Community Page}
\\Inspired by the Chinese music app NetEase, one of our goals is to create a community page for our users to share their works and collaborate, 
we believe that the community element can bring continuous traffic to our app, which can benefit the monetisation of our app in the future.
As we mentioned before, the user experience determines the success of the app design. Here, the quality of the posts recommended to our users is an essential element affecting our user experience. 
Therefore, it requires our recommendation engine to be well-designed. Chapter \ref{Chapter6}

\item \textbf{Message Channel}
\\The message channel allows our users to contact each other directly and share the original chord files to collaborate on.

\end{itemize}

%-----------------------------------------------------------------------------------------------------------------------------UI
\section{UI Design}
After having all the user flow maps of our design, we started to design the prototype using Figma, a design tool for prototyping projects. 
There are three example sketches in \autoref{fig:UIdesign}. 

\begin{figure}[ht]
     \centering
     \hspace{16mm}
     \begin{subfigure}[b]{0.2\textwidth}
         \centering
         \includegraphics[width=\textwidth]{mianpage1.png}
         \caption{Main page - audio-upload page}
         \label{Mainpage}
     \end{subfigure}
     \hfill
     \begin{subfigure}[b]{0.2\textwidth}
         \centering
         \includegraphics[width=\textwidth]{grappp.png}
         \caption{Chord layout and editing page}
         \label{chordedit}
     \end{subfigure}
     \hfill
     \begin{subfigure}[b]{0.2\textwidth}
         \centering
         \includegraphics[width=\textwidth]{commupage.png}
         \caption{Community page}
         \label{Community page}
     \end{subfigure}
     \hspace{16mm}
        \caption{UI design for our app}
        \label{fig:UIdesign}
\end{figure}

\section{Testing}
The testing stage allows us to discover the problems in our current designs and improve our existing designs. 
According to Compuware, 48\% of users are less likely to use an app again if they are troubled with its performance.
As reported by Compuware, only 16\% of users can give the app a try for a second or third time. \footfullcite{lowtole}
\\
We divide our testing stage into four stages, Unit Tests, Integration Tests, System Tests, and Acceptance Tests.
\begin{itemize}
\item\textbf{Unit Test} is the very first stage of any application test. Here, the system’s separate modules will undergo assessments to see if they, individually, function correctly and to maximum capacity. 
\item\textbf{Integration Test} level is mostly about verifying the modules and checking their readiness and their collective, integral cooperation. The modules are each tested separately and also as a group. This aids the testers in identifying any issues with two or more components working together or individually to execute functions.
\item\textbf{System Test} level closely simulates the final production environment. It is very significant in the final testing stage, especially when it needs to ensure that the applications under scrutiny always meet complete functional requirements. 
This is where the test team ascertains if the integrated components are collectively showing optimal performance levels or not. For this, every software build must undergo testing up to this point, using client requirements as a benchmark. 
This stage is carried out by the quality assurance engineer in our team.
\item\textbf{Acceptance Test} is the stage where we will include A/B testing in our Beta test. After we have a functionally-reliable version of the app, 
we will be performing A/B testing by providing different UI designs to other groups of test users; this method allows us to 
determine which option is more attractive.
\end{itemize}


